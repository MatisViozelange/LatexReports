\chapter{Introduction}

This report documents my international mobility experience as part of my 
TFE at Centrale Nantes. The internship was conducted at Technische 
Universität Ilmenau, located in the heart of Thuringia, Germany and led 
by my supervisor PhD. student Niclas Tietze. 
Known for its rigorous academic environment and focus on applied sciences, 
TU Ilmenau provided an ideal setting for my research project.

The Institut für Automatisierungs und Systemtechnik at Technische 
Universität Ilmenau is a dynamic research group specializing in 
advanced control systems and automation technologies. Led by Prof. 
Johann Reger, the team comprises 14 members, including professors, 
researchers, and PhD students, all contributing to various projects in 
the field of control engineering. The primary objective of my internship 
was to work on the project titled Resolution of algebraic loops 
in sliding mode control approaches. This project aimed to address the 
challenges posed by algebraic loops in sliding mode control algorithms, 
particularly focusing on state-dependent disturbances that create circular 
dependencies with the control signal which complicate the already known
stability proofs for controller of the super-twisting type.

During my internship, I also had the opportunity to study and implement 
the Model Following Control (MFC) strategy, which is a robust control 
technique designed to ensure that a system's output follows a desired 
reference model, even in the presence of uncertainties and disturbances. 
MFC is particularly useful in applications requiring precise tracking of 
a reference trajectory, such as in robotics, aerospace, and automotive 
systems. One of the key challenges addressed by MFC is the peaking 
phenomenon, which occurs in high gain control systems and can lead to 
large, temporary deviations in the system's response. The MFC strategy 
mitigates this phenomenon by using a two-loop structure: the Model Control 
Loop (MCL) for nominal control and the Process Control Loop (PCL) for 
perturbation compensation.

Throughout this report will be provided a first part on the MFC strategy, the 
theoretical background then a focus on its assets in both implementation and 
performance compared to high gain control which is very close to MFC.
Next will be a second part on the algebraic loops resolution in sliding mode
control approaches, with the context of the problem, the state of the art
and the methodology used to solve this problem. A method will be proposed 
for a certain class of perturbation and then clues and ideas for the
general case will be given.

This internship has not only enhanced my technical knowledge in control 
systems but also provided me with a deeper understanding of German culture 
and academic practices. I am grateful for the opportunity to have been a 
part of such a dynamic and innovative research team.
