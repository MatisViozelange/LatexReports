

\chapter{Local Stability of high order super-twisting with time and state dependent perturbations}

\section{Introduction} 

This chapter presents the other problem I had to work on during my internship. It is based on the articles by my supervisor \cite{tietze2024local}, \cite{tietze2024localStabilisation}, and \cite{tietze2025dynamic}. The work starts from the reflection that in the proof given in the previous work on the sliding mode algorithm \cite{Moreno2012}, the perturbation is assumed to be bounded and only time dependent. This is a strong assumption, as in many practical cases, the perturbation can depend on the state of the system which leads to the creation of an algebraic loop. Let's consider the r-th order perturbed chain of integrators:

\begin{equation}
    \label{eq:perturbed_chain}
    (S) \qquad \dot{z} = J_r z + (\Delta(t,z) + u) e_r, \quad z = (z_1, z_2, \ldots, z_r) \in \mathbb{R}^r, \; u \in \mathbb{R}
\end{equation}

where $J_r$ is the r-th Jordan block, $e_r = (0, \ldots, 0, 1)^T \in \mathbb{R}^r$ and 
$\Delta: \mathbb{R}_+ \times \mathbb{R}^r \to \mathbb{R}$ is a perturbation. The goal is to design a 
feedback law $u = u(t,z)$ such that the closed-loop system is finite time stable with respect to the origin.

The work initiated by \cite{tietze2024local}, \cite{tietze2024localStabilisation} and 
\cite{tietze2025dynamic} took different variations of the Sliding Mode control (SMC) algorithm and sucessfully
proved the local finite time stability of the closed loop system. Here we are looking for the 
same kind of proof on the same model but for the High Order Super Twisting algorithm (HOST).

Several work already exist on he stability of HOST with time and state dependent perturbation, like 
\cite{Moreno2012} for the classical Super Twisting algorithm or \cite{kamal2014higher} which extend the
case the the r-th order. However, the proof is not complete as it does not take into account the
algebraic loop created by the state dependence of the perturbation.

The results presented in \cite{Laghrouche2017} on a different kind of HOST algorithm, gives a proof
of global finite time convergence with the HOST defined in \cite{Hong2005}. However the proof is
not done on a time and state dependent perturbation and uses a different strategy than the one used in
\cite{khalil2002nonlinear} based on invariant sets constrained by a Lyapunov function.


\newpage
\section{Context and problem statement}

We consider the perturbed chain of integrators defined in \eqref{eq:perturbed_chain} with a perturbation
$\Delta(t,z)$ satisfying the following assumption:

\begin{equation}
    \label{eq:assumption_perturbation_sum}
    \Delta(t,z) = \Delta_t(t) + \Delta_z(z)
\end{equation}

where $\Delta_t: \mathbb{R}_+ \to \mathbb{R}$ is a time dependent perturbation and 
$\Delta_z: \mathbb{R}^r \to \mathbb{R}$ is a state dependent perturbation. We assume that the state \(z(t)\) 
is always available for control. The initial value of the system is denoted by \(z_0 = z(0) \in \mathbb{R}^r\). 

The Goal is to design a feedback law \(u = u_{st}(t,z)\) such that the closed-loop system is finite time stable
with respect to the origin \(z = 0\).

First, we will analyse the control law given in \cite{Laghrouche2017} and \cite{Hong2005} and their stability
proof. Then we will try to adapt this vision to our problem and see if we can get to this result by 
taking into account the state dependent perturbation.

\subsection{High Order Super Twisting algorithm}

First of all, let's define the HOST algorithm as in \cite{Laghrouche2017}.

The following uses a results on stabilization theory on \ref{eq:perturbed_chain} as well as a geometric condition on the
homogeneous stabilizing feedback.

Let \(\mathcal{K} < 0\) and \(p > 0\) with \( p + (r + 1)\kappa \geq 0 \), set 
\( \pi_i := p + (i - 1)\kappa \), \( 1 \leq i \leq r + 1 \). For \( \epsilon > 0 \), let 
\( \delta_{\epsilon} : \mathbb{R}^r \rightarrow \mathbb{R}^r \) and \( \psi_{\epsilon} :
 \mathbb{R}^{r+1} \rightarrow \mathbb{R}^{r+1} \) be the family of dilations associated 
 with \( (p_1, \ldots, p_r) \) and \( (p_1, \ldots, p_{r+1}) \) respectively.


\begin{Proposition}
\label{prop:stabilizing_feedback}
Let \( r \) be a positive integer. There exists a feedback law \( u_0 : \mathbb{R}^r \to \mathbb{R} \), 
homogeneous of degree \( p_r + 1 \) with respect to \( (\delta_\epsilon)_{\epsilon > 0} \) such that the 
closed-loop system (CI)\textsuperscript{r} with \( u_0 \) is finite time stable and the following conditions 
hold true:

\begin{enumerate}
    \item The function \( z \mapsto Jr z + u_0(z) e_r \) is homogeneous of degree \( \kappa \) with 
    respect to \( (\delta_\epsilon)_{\epsilon > 0} \), and there exists a continuous positive definite 
    function \( V_1: \mathbb{R}^r \to \mathbb{R}_+ \), \( C^1 \) except at the origin, homogeneous with 
    respect to \( (\delta_\epsilon)_{\epsilon > 0} \) of degree \( 2p_r + 1 \) such that there exists 
    \( c > 0 \) and \( \alpha \in (0, 1) \) for which the time derivative of \( V_1 \) along non-trivial 
    trajectories of (CI)\textsuperscript{r} verifies \( \dot{V}_1 \leq -cV_1^\alpha \).
    \item \( z \mapsto \partial_r V_1(z) \) is homogeneous of positive degree with respect 
    to \( (\delta_\epsilon)_{\epsilon > 0} \) and \( z \mapsto \partial_r V_1(z)u_0(z) \) is 
    negative over \( \mathbb{R}^r \).
\end{enumerate}


\end{Proposition}


We will take \( \alpha \) equal to : \( 1 + \frac{\kappa}{2p_r + 1} = \frac{1}{2} \) to get homogeneity properties
for a Lyapunov function to be used in the proof and a condition that assure finite time convergence.

\( \partial_r V_1 \) is homogeneous with respect to \( (\delta_\epsilon)_{\epsilon > 0} \) of 
degree \( p + (r + 1)\kappa = 0 \). In \cite{Hong2005}, a feedback law \( u_0: \mathbb{R}^r \to \mathbb{R} \) 
satisfying Condition (1) of Prop.\ref{prop:stabilizing_feedback} is explicitly built.

Now we can define the HOST algorithm as follows:

\begin{theorem}
    Consider the homogeneous mapping $u_0 : \mathbb{R}^r \rightarrow \mathbb{R}$ and the continuous positive 
    definite function $V_1 : \mathbb{R}^r \rightarrow \mathbb{R}_+$ provided by Proposition 1. For every 
    $k_P \geq 1$ and $k_I > 0$, let $u_{ST}(\cdot)$ be the dynamic state-feedback controller defined by

    \begin{align}
        u_{ST}(t) &= k_P u_0(z(t)) + k_I \xi(t) \label{eq:u_ST} \\
        \dot{\xi}(t) &= -k_I \partial_r V_1(z(t)), \quad \xi(0) = 0 \label{eq:xi_dot} 
    \end{align}

    and we refer to $u_{ST}$ as the \textit{HOST} controller. The feedback connection between 
    \ref{eq:perturbed_chain} with the assumption \(\Delta(t, z) = 0\) and \eqref{eq:u_ST} gives rise to the 
    dynamical system over $\mathbb{R}^{r+1}$ given by

    \begin{align}
        \dot{z}(t) &= J_r z(t) + (k_P u_0(z(t)) + k_I \xi(t)) e_r \label{eq:z_dot} \\
        \dot{\xi}(t) &= -k_I \partial_r V_1(z(t)), \quad \xi(0) = 0. \label{eq:xi_dot_system}
    \end{align}

    There exist $A, d > 0$ such that, if $W : \mathbb{R}^{r+1} \rightarrow \mathbb{R}$ is defined by

    \begin{equation}
        W(z, \xi) = \left(V_1(z) + \frac{\xi^2}{2}\right)^{2-\alpha} - A z_r \xi, \label{eq:W}
    \end{equation}

    then $W$ is positive definite, $C^1$ except at the origin, homogeneous with respect 
    to $(\psi_\varepsilon)_{\varepsilon > 0}$ and the time derivative of $W$ along non-trivial 
    trajectories of \eqref{eq:z_dot}-\eqref{eq:xi_dot_system} verifies $\dot{W} \leq -d W^{1/(2-\alpha)}$. 
    As a consequence, trajectories of \eqref{eq:z_dot}-\eqref{eq:xi_dot_system} converge to zero in finite 
    time, i.e., $u_{ST}$ stabilizes \ref{eq:perturbed_chain} in finite time.
\end{theorem}


This theorem gives a first framework for the stability proof of the HOST algorithm. Also we can notice that 
the control law and the condition on A and d are also given in \cite{Laghrouche2017}. The complete proof 
is not given in \cite{Laghrouche2017} but can be found in the appendix\ref{app:proof_theorem1} with 
the explicit values of A and d.

The next step in this work is to introduce the perturbation \(\Delta(t,z)\) in the closed loop system 
and see how it can be taken into account in the stability proof. With the assumption that the perturbation
only depends on time \(\Delta(t,z) = \Delta_t(t)\), and it is bounded with : 

\begin{equation}
    \label{eq:assumption_perturbation_time_forLagrouche}
    |\Delta_t(t)| \leq \delta, \quad |\dot{\Delta}_t(t)| \leq \delta_t, \quad \forall t \geq 0
\end{equation}

The solution given in \cite{Laghrouche2017} is to introduce a time scaling factor and by a change of
variable and by using the homogeneity properties of the system, we can prove the finite time stability of the
closed loop system. By considering the next theorem:

% \begin{theorem}
%     Consider the perturbed chain of integrators defined by \ref{eq:perturbed_chain}, where 
%     the perturbation \(\Delta\) assures the assumption \ref{eq:assumption_perturbation_time_forLagrouche}. 
%     Assume that there exists a continuous homogeneous feedback law \(u_0\) and a Lyapunov function \(V_1\) 
%     verifying the assumptions (1) and (2) of Proposition \ref{prop:stabilizing_feedback} with \(p+(r+1)\kappa=0\). 
%     Then, for every positive gains \(k_P \geq 1\) and \(k_I > 0\), there exists \(\lambda_0 > 0\) only 
%     depending on the gains and \(\delta_t\) such that, for \(\lambda \geq \lambda_0\), the dynamic 
%     state-feedback controller \(u_{ST}^{\lambda}(\cdot)\) defined by

%     \begin{align}
%         \label{eq:u_ST_lambda}
%         u_{ST}^{\lambda}(t) &= \left(k_P u_0(D_{\lambda} z(t)) + k_I \xi^{\lambda}(t)\right) \\
%         \dot{\xi}^{\lambda}(t) &= -\lambda k_I \partial_r V_1(D_{\lambda} z(t)), \quad \xi^{\lambda}(0) = 0
%     \end{align}
    
%     where \(D_{\lambda} = \text{diag}(\lambda^{r-1}, \ldots, \lambda, 1)\) and stabilizes \ref{eq:perturbed_chain} in finite-time. In particular, \(u_{ST}^{\lambda}(\cdot)\) is continuous.
% \end{theorem}

% Fix \(k_P \geq 1\) and \(k_I > 0\). For every \(\lambda > 0\), consider the standard time-coordinate 
% change of variable along trajectories of \ref{eq:perturbed_chain} defined by \(y(t) = D_{\lambda} z(t/\lambda)\). 
% Under the hypotheses of the theorem, \ref{eq:perturbed_chain} can be rewritten as 

% \begin{equation}
%     \label{eq:perturbed_chain_scaled}
%     \dot{y} = J_r y + (u_{\lambda} + \Delta_{\lambda}) e_r
% \end{equation}

% where one has set, for \(t \geq 0\), \(u_{\lambda}(t) := u(t/\lambda)\) and \(\Delta_{\lambda}(t) := \Delta(t/\lambda)\) with \(|\dot{\Delta}_{\lambda}| \leq \Delta_t/\lambda\).

% The feedback connection between \ref{eq:perturbed_chain} and \ref{eq:u_ST_lambda} can be written as the 
% time-varying system over \(\mathbb{R}^{r+1}\) given by

% \begin{align}
%     \label{eq:closed_loop_perturbed_scaled}
%     \dot{y}(t) &= J_r z(t) + \left(k_P u_0(y(t)) + k_I \xi_{\lambda}(t)\right) e_r \\
%     \dot{\xi}_{\lambda}(t) &= -k_I \partial_r V_1(y(t)) + \dot{\Delta}_{\lambda}, \quad \xi_{\lambda}(0) = \Delta_{\lambda}(0)
% \end{align}

% where \(\xi_{\lambda}(t) = \xi^{\lambda}(t/\lambda) + \Delta_{\lambda}(t)\).

% Clearly, \ref{eq:closed_loop_perturbed_scaled} corresponds to the differential 
% inclusion \ref{eq:z_dot}-\ref{eq:xi_dot_system} perturbed by the time-varying vector field over 
% \(\mathbb{R}^{r+1}\) given by \((0, \cdots, 0, \dot{\Delta}_{\lambda}(t))^T\) or, equivalently, by 
% the multifunction : 

% \begin{equation}
%      (0, \cdots, 0, [-\Delta_t/\lambda, \Delta_t/\lambda])^T
% \end{equation}

% ,taking values in the subsets of \(\mathbb{R}^{r+1}\). Let \(W\) be the Lyapunov function defined in \ref{eq:W}. 
% Along non trivial trajectories of \ref{eq:closed_loop_perturbed_scaled}, one gets, for every \(t \geq 0\), that : 

% \begin{equation}
%     \dot{W} \leq -dW^{2/3} + \partial_{\xi}W \dot{\Delta}_{\lambda}(t) \leq -dW^{2/3} + \Delta_t|\partial_{\xi}W|/\lambda
% \end{equation}

% According to Remark 1, the homogeneity degree of \(|\partial_{\xi}W|\) is equal to 
% \(3p_{r+1} - p_{r+1} = 2p_{r+1}\), i.e., the homogeneity degree of \(W^{2/3}\). One deduces from the 
% previous equation that there exists \(\Delta_* > 0\) such that \(\dot{W} \leq -dW^{2/3}/2\), along 
% trajectories of System \ref{eq:closed_loop_perturbed_scaled} if \(\Delta_t/\lambda \leq \Delta_*\), i.e., 
% \(\lambda \geq \lambda_0 = \Delta_t/\Delta_*\). Hence the conclusion.


% The contents of Theorem 2 can be derived without relying on HOST controllers but rather on extending the 
% relative degree of the original system. The latter techniques, however, require a longer chain of integrators. 
% Notice that the choice of \(\lambda_0\) can be made explicit once \(u_0\) and \(V_1\) are explicitly given. 
% We do not know how to extend the above result to cases where \(p+(r+1)\kappa>0\). Indeed, for the above 
% homogeneity argument to work, it is necessary that \(W^{1/(2-\alpha)}\) (with \(\alpha=1/2\)) has the same 
% degree of homogeneity as \(\partial_{\xi}W\) since \(\dot{\Delta}\) is simply bounded. On the other hand, 
% \(W^{1/(2-\alpha)}\) has the same degree of homogeneity as \(\partial_{\xi}W \xi\) and thus \(\partial_r V_1\) 
% must necessarily be of degree zero. This occurs only if \(p+(r+1)\kappa=0\).

% \subsection{Modified Theorem}

Consider the perturbed chain of integrators defined by Equation \ref{eq:perturbed_chain}, where the 
perturbation \(\Delta\) satisfies assumption \ref{eq:assumption_perturbation_time_forLagrouche}.

Assume there exists a continuous homogeneous feedback law \(u_0\) and a Lyapunov 
function \(V_1\) satisfying the following assumptions (1) and (2) of Proposition 
\ref{prop:stabilizing_feedback} with \(p+(r+1)\kappa=0\):

\begin{enumerate}
    \item The function \( z \mapsto Jr z + u_0(z) e_r \) is homogeneous with respect 
    to \((\delta_\epsilon)_{\epsilon > 0}\), and there exists a continuous positive definite 
    function \( V_1: \mathbb{R}^r \to \mathbb{R}_+ \), \( C^1 \) except at the origin, homogeneous 
    with respect to \((\delta_\epsilon)_{\epsilon > 0}\).

    \item \( z \mapsto \partial_r V_1(z) \) is homogeneous of positive degree with respect 
    to \((\delta_\epsilon)_{\epsilon > 0}\) and : \begin{equation} z \mapsto \partial_r V_1(z)u_0(z) \end{equation} is negative 
    over \(\mathbb{R}^r\).
\end{enumerate}

We have the following theorem:

\begin{theorem}
    For every positive gains \(k_P \geq 1\) and \(k_I > 0\), there exists \(\lambda_0 > 0\) depending only 
    on the gains and \(\delta_t\) such that, for \(\lambda \geq \lambda_0\), the dynamic state-feedback 
    controller \(u_{ST}^{\lambda}(\cdot)\) defined by

    \begin{align}
        u_{ST}^{\lambda}(t) &= \left(k_P u_0(D_{\lambda} z(t)) + k_I \xi^{\lambda}(t)\right) \label{eq:u_ST_lambda} \\
        \dot{\xi}^{\lambda}(t) &= -\lambda k_I \partial_r V_1(D_{\lambda} z(t)), \quad \xi^{\lambda}(0) = 0
    \end{align}

    where \(D_{\lambda} = \text{diag}(\lambda^{r-1}, \ldots, \lambda, 1)\), stabilizes \ref{eq:perturbed_chain} 
    in finite-time. In particular, \(u_{ST}^{\lambda}(\cdot)\) is continuous.
\end{theorem}

Fix \(k_P \geq 1\) and \(k_I > 0\). For every \(\lambda > 0\), consider the standard time-coordinate change 
of variable along trajectories of \ref{eq:perturbed_chain} defined by \(y(t) = D_{\lambda} z(t/\lambda)\). 
Under the theorem's hypotheses, \ref{eq:perturbed_chain} can be rewritten as:

\begin{equation}
    \dot{y} = J_r y + (u_{\lambda} + \Delta_{\lambda}) e_r
\end{equation}

where for \(t \geq 0\), \(u_{\lambda}(t) := u(t/\lambda)\) and \(\Delta_{\lambda}(t) := \Delta(t/\lambda)\) with \(|\dot{\Delta}_{\lambda}| \leq \Delta_t/\lambda\).

The feedback connection between \ref{eq:perturbed_chain} and \ref{eq:u_ST_lambda} can be written as the 
time-varying system over \(\mathbb{R}^{r+1}\) given by:

\begin{align}
    \dot{y}(t) &= J_r z(t) + \left(k_P u_0(y(t)) + k_I \xi_{\lambda}(t)\right) e_r \\
    \dot{\xi}_{\lambda}(t) &= -k_I \partial_r V_1(y(t)) + \dot{\Delta}_{\lambda}, \quad \xi_{\lambda}(0) = \Delta_{\lambda}(0)
    \label{eq:closed_loop_perturbed_scaled}
\end{align}

where \(\xi_{\lambda}(t) = \xi^{\lambda}(t/\lambda) + \Delta_{\lambda}(t)\).

Clearly, \ref{eq:closed_loop_perturbed_scaled} corresponds to the differential 
inclusion \ref{eq:z_dot}-\ref{eq:xi_dot_system} perturbed by the time-varying vector field 
over \(\mathbb{R}^{r+1}\) given by \((0, \cdots, 0, \dot{\Delta}_{\lambda}(t))^T\) or, equivalently, 
by the multifunction:

\begin{equation}
     (0, \cdots, 0, [-\Delta_t/\lambda, \Delta_t/\lambda])^T
\end{equation}

Let \(W\) be the Lyapunov function defined in \ref{eq:W}. Along non-trivial trajectories 
of \ref{eq:closed_loop_perturbed_scaled}, one gets, for every \(t \geq 0\), that:

\begin{equation}
    \dot{W} \leq -dW^{2/3} + \partial_{\xi}W \dot{\Delta}_{\lambda}(t) \leq -dW^{2/3} + \Delta_t|\partial_{\xi}W|/\lambda
\end{equation}

According to Remark 1 of \cite{Laghrouche2017}, the homogeneity degree of \(|\partial_{\xi}W|\) is equal 
to \(3p_{r+1} - p_{r+1} = 2p_{r+1}\), i.e., the homogeneity degree of \(W^{2/3}\). One deduces from 
the previous equation that there exists \(\Delta_* > 0\) such that \(\dot{W} \leq -dW^{2/3}/2\), along 
trajectories of System \ref{eq:closed_loop_perturbed_scaled} if \(\Delta_t/\lambda \leq \Delta_*\), 
i.e., \(\lambda \geq \lambda_0 = \Delta_t/\Delta_*\).

The contents of Theorem 2 can be derived without relying on HOST controllers, instead extending the 
relative degree of the original system. The latter approach, however, requires a longer chain of integrators.

Notice that the choice of \(\lambda_0\) can be made explicit once \(u_0\) and \(V_1\) are explicitly 
given. We do not know how to extend the above result to cases where \(p+(r+1)\kappa>0\). Indeed, for the 
above homogeneity argument to work, it is necessary that \(W^{1/(2-\alpha)}\) (with \(\alpha=1/2\)) has the 
same degree of homogeneity as \(\partial_{\xi}W\) since \(\dot{\Delta}\) is simply bounded. On the other 
hand, \(W^{1/(2-\alpha)}\) has the same degree of homogeneity as \(\partial_{\xi}W \xi\) and 
thus \(\partial_r V_1\) must necessarily be of degree zero. This occurs only if \(p+(r+1)\kappa=0\).

One remark we can make on the dilatation function and the necessary homogeneity of certain functions in the
assumptions is mandatory to get the conditions on A and d in the proof of Theorem 1 to extend the result
on the unit ball to a global stability with finite time convergence.

We now want to extend this result to the case where the perturbation is time and state dependent like in 
\ref{eq:assumption_perturbation_sum}. Unfortunately, the time scaling argument does not work in this case
because the term \(\lambda\) has no effect on the state dependent part of the perturbation derivative as 
we can see below:

\begin{equation}
    \dot{\Delta}(t) = \frac{\partial \Delta_t(t)}{\partial t} + \frac{\partial \Delta_t(t)}{\partial z} \dot{z}(t)
\end{equation}

so we got : 

\begin{equation}
    \dot{\Delta}_{\lambda}(t) = \frac{1}{\lambda} \frac{\partial \Delta_t(t)}{\partial t} + \frac{\partial \Delta_z(z(t))}{\partial z} \dot{z}
\end{equation}

It is easy to see that the second term is not affected by \(\lambda\) and thus we cannot use the same argument 
as before.

Now let's have a look on the work done in \cite{tietze2025dynamic} where the approach is different and deals 
with algebraic loops but only for a SMC Super Twisting algorithm.






















\subsection{The Super Twisting case with time and state dependent perturbation}

Now we will present the work done in \cite{tietze2025dynamic} on the Super Twisting algorithm with time and state
dependent perturbation. The goal is to estabilish a local stability result for the Super Twisting algorithm (STA)
and use the mathematical tools to try to get the same kind of result for the HOST algorithm.

The constrained sets with Lyapunov functions used in this work is also presented in \cite{khalil2002nonlinear}. 
Let's consider the following sections.


\subsubsection{Stability for Unbounded Perturbations : Sliding Mode case}

To construct an invariant measure for the functional \( V \), let
\(z = \begin{bmatrix} z_1 & z_2 \end{bmatrix}^T \in \mathbb{R}^n\) and \(\omega \in \mathbb{R}^n_\omega\).

We define the sliding variable:

\begin{align}
    s(z, \omega) &= Lz_1 + z_2 + H\omega \\
    \dot{\omega} &= Gz_1 + F\omega
\end{align}

With \( z_2 = s(z, \omega) - Lz_1 - H\omega \)

The dynamics are described by:

\begin{equation}
    \begin{cases}
        \dot{z}_1 = (A - BL)z_1 - BH\omega + Bs(z, \omega) \\
        \dot{s} = -\rho \text{sgn}(s(z, \omega)) + \Delta(z, t) \\
        \dot{\omega} = Gz_1 + F\omega
        \label{eq:closed_loop_dynamicNiclas}
    \end{cases}
\end{equation}

We consider the following Lyapunov function:

\begin{equation}
    V(z_1, \omega) = [z \quad \omega]^T P [z \quad \omega],
    \label{eq:Lyapunov_V}
\end{equation}

where \( P = P^T \geq 0 \) and \( A_{cl}^T P + P A_{cl} = -I \).
We also have:
\begin{equation}
    a = 2 \lambda_{\min}(P) \|P B\|_2.
\end{equation}
For a given \( c \geq 0 \), we define:

\begin{equation}
    \Omega_{c, c_\omega} = \left\{ \begin{pmatrix} z \\ \omega \end{pmatrix} \mid \|s(z, \omega)\| \leq c \text{ and } V(z, \omega) \leq c_\omega^2 \right\},
\end{equation}

where \( c_\omega > a c \).

We consider the following projection of the set \( \Omega_{c, c_\omega} \) on the z subset :

\begin{equation}
    \Psi_{c, c_\omega} = \left\{ z \mid \exists \omega \in \mathbb{R}^n_{\omega}, \begin{pmatrix} z \\ \omega \end{pmatrix} \in \Omega_{c, c_\omega} \right\}.
\end{equation}

Now, all the sets and condition are defined, we can state the main assumption and theorem of this section.

\begin{assumption}
    \label{assumption:H4}
    For given \( c \geq 0 \) and \( c_\omega \geq a c \), there exists \(\delta > 0\) such that 
    for all \(t \geq 0 \) and \( z \in \Psi_{c, c_\omega} \), then \( |\Delta(z, t)| \leq \delta \).
\end{assumption}


\begin{theorem}
    Consider the closed loop \ref{eq:closed_loop_dynamicNiclas} and assumption \ref{assumption:H4} 
    Given \( z(0) = z_0 \) and \( \omega(0) = \omega_0 \), with \( \rho > \delta \),
    Then:
    For all \((x_0, \omega_0) \in \Omega_{c, c_\omega}\), the solution \((z, \omega)\) is bounded and,
    \begin{equation}
        \begin{cases}
            &\forall t \geq 0, (z(t), \omega(t)) \in \Omega_{c, c_\omega} \\
            &\lim_{t \to \infty} (z(t), \omega(t)) \to 0 \\
            &s(z, \omega) = 0, \quad \forall t > t_0 = \min \{ t \geq 0 \mid s(z(t), \omega(t)) = 0 \}
        \end{cases}
    \end{equation}
\end{theorem}

This proof is very classical but gives a good framework for what is comming next with the introduction of the 
algebraic loop in the STA. The new assumption we are going to make only hold locally in the invariant sets 
and the theorem will be contructed on the same method.

The proof is given in the appendix \ref{app:ProofSMCLocalStabNiclas}.

\subsection{Stability for Unbounded Perturbations : Super Twisting case}

We have to consider the perturbation as the sum like in \ref{eq:assumption_perturbation_sum} 
: \( \Delta(z, t) = \Delta_z(z) + \Delta_t(t) \)

\textbf{Control Design} 

We use the classical STA with a gain scaling \(\mu > 0\) to compensate the perturbation:
\begin{equation}
    \begin{cases}
        u = u_0 - \mu^{-1} \alpha_1 |s(z, \omega)|^{1/2} \text{sgn}(s(z, \omega)) + v \\
        \dot{v} = -\frac{1}{2} \mu^{-2} \alpha_2 \text{sgn}(s(z, \omega)), \quad v(0) = v_0
    \end{cases}
\end{equation}

With the gains \( \alpha_1, \alpha_2 > 0 \), and the gain scale: \( \mu > 0 \).

This leads to the new closed loop dynamics:
\begin{equation}
    \begin{cases}
        \dot{z} = (A - BL)z_1 - BH\omega + B s(z, \omega) \\
        \dot{s} = -\mu^{-1} \alpha_1 |s(z, \omega)|^{1/2} \text{sgn}(s(z, \omega)) + v + \Delta(z, t) \\
        \dot{v} = -\frac{1}{2} \mu^{-2} \alpha_2 \text{sgn}(s(z, \omega)) \\
        \dot{\omega} = Gz_1 + F\omega
    \end{cases}
    \label{eq:closed_loop_dynamicNiclasSTA}
\end{equation}

\textbf{Stability of the Closed Loop} 

We have to create a new invariant set \(\Lambda_{c, c_\omega, \mu}\) for the extended state :
\((z, \omega, v) \in \mathbb{R}^{n + n_\omega + 1}\).

Given \( \alpha_1, \alpha_2 > 0 \), we note:

\begin{equation}
    A_s = \frac{1}{2} \begin{bmatrix} -\alpha_1 & 1 \\ -\alpha_2 & 0 \end{bmatrix}, \quad B_s = \begin{bmatrix} 0 \\ 1 \end{bmatrix}, \quad P_s = \begin{pmatrix} p_{11} & p_{12} \\ p_{12} & p_{22} \end{pmatrix}
\end{equation}

The solution of \( A_s^T P_s + P_s A_s = -I \) with \( p_{12} = -1 \).
Consider the following objective function:

\begin{equation}
V_\mu(s, v) = p_{11} |s| + 2 p_{12} \mu v |s|^{1/2} \text{sgn}(s) + p_{22} \mu^2 v^2
\end{equation}

We introduce also the compact set:

\begin{equation}
\Gamma_{c, \mu} = \left\{ \begin{pmatrix} s \\ v \end{pmatrix} \mid V_\mu(s, v) \leq k(c) \right\}
\end{equation}

where

\begin{equation}
k(c) = \frac{p_{11} p_{22} - p_{12}^2}{p_{22}} c
\end{equation}

We can show:
For \(\begin{pmatrix} s \\ v \end{pmatrix} \in \Gamma_{c, \mu}\), \(|s| \leq c\) and 
\(|v| \leq \frac{1}{\mu} \sqrt{\frac{p_{11}}{p_{22}}c}\).
Note: The upper bound of \(|v|\) increases when the value of \(\mu\) decreases and vice versa.
Therefore, for \(\forall \epsilon \in \mathbb{R}\):

\begin{equation}
    \exists v \in \mathbb{R}, \left( \begin{pmatrix} s \\ v \end{pmatrix} \in \Gamma_{c, \mu} \right) \iff |s| \leq c
\end{equation}

We consider the following Lyapunov function candidates \(V_\mu\) and \(V(z_1, \omega)\) \ref{eq:Lyapunov_V} 
and the scalar:
\begin{equation}
    a = 2 \lambda_{\max}^{1/2}(P) \|PB\|_2
\end{equation}

We define the following set:
\begin{equation}
    \Lambda{c, c_\omega, \mu} = \left\{ \begin{pmatrix} z \\ \omega \\ v \end{pmatrix} \mid V_\mu(s(z, \omega), v) \leq k(c) \cap V(z, \omega) \leq c_\omega^2 \right\}
\end{equation}

And their projections in the subspaces $z-\omega$ and $z$:

\begin{equation}
    \Phi_{c, c_\omega} = \left\{ \begin{pmatrix} z \\ \omega \end{pmatrix} \mid \exists v \in \mathbb{R}, \begin{pmatrix} z \\ \omega \\ v \end{pmatrix} \in \Lambda_{c, c_\omega, \rho} \right\}
\end{equation}
\begin{equation}
    \Psi_{c, c_\omega} = \left\{ z \mid \exists (\omega, v) \in \mathbb{R}^{m_\omega + 1}, \begin{pmatrix} z \\ \omega \\ v \end{pmatrix} \in \Lambda_{c, c_\omega, \rho} \right\}
\end{equation}

\begin{lemma}
The projections \(\Phi_{c, c_\omega}\) and \(\Psi_{c, c_\omega}\) don't depend on \(\mu > 0\) and 
\(\Phi_{c, c_\omega} = \Omega_{c, c_\omega}\) and
\(\Psi_{c, c_\omega} = \Psi_{c, c_\omega} \quad \forall c > 0 \text{ or } c_\omega > a c\).
\end{lemma}

Given a variable of comparison \(u_c\), \(\exists u_{\max} > 0\) independent of \(\mu\) such that:
\begin{equation}
    \|Az_1 + Bz_2\|_2 + |u_0| \leq u_{\max} \quad \forall \begin{pmatrix} z \\ \omega \end{pmatrix} \in \Phi_{c, c_\omega}
\end{equation}

\begin{assumption}
    For all \(c \geq 0\) and \(c_\omega \geq a c\), there exist \(\delta, \delta_t, \delta_z > 0\) such that 
    for all \(t \geq 0\) and all \(z \in \Phi_{c, c_\omega}\), we have:
    \begin{equation}
        |\Delta(z, t)| \leq \delta,
    \end{equation}
    \begin{equation}
        \left| \frac{d\Delta_t(t)}{dt} \right| \leq \delta_t,
    \end{equation}
    \begin{equation}
        \left| \frac{\partial \Delta_z(z)}{\partial z} \right| \leq \delta_{z} 
    \end{equation}
\end{assumption}

These lemma and assumption give the key to have conditions on the gain \(\mu\) to ensure the stability 
in case of unbounded perturbation. We can proove the next theorem by the same method as before with the 
invariant sets and the Lyapunov functions we created in this section. The proof can be found in \cite{tietze2025dynamic}.


\begin{theorem}
    Consider the Closed Loop of \ref{eq:closed_loop_dynamicNiclasSTA} and the Initial Conditions:

    \begin{equation} 
        z(0) = z_0, \quad \omega(0) = \omega_0, \quad v(0) = v_0 
    \end{equation}

    Let

    \begin{equation}
        \mu_0 = (\delta \|P_0 B_0\|_2)^{-1} \left( \frac{k(c)}{\lambda_{\min}(P_s)} \right)^{1/2},
    \end{equation}

    \begin{equation}
        \mu_1 = (2\gamma \|P_s B_s\|_2)^{-1}.
    \end{equation}

    For \(\gamma = \mu_0 (\delta_t + \delta_z U_{\max} + \delta_z \delta) + \delta_z (\alpha_1 + \sqrt{\frac{p_{11}}{p_{22}}})\sqrt{c}\),
    
    For all \(\mu < \min(\mu_0, \mu_1)\), and \(\begin{pmatrix} z_0 \\ \omega_0 \\ v_0 \end{pmatrix} \in \Lambda_{c, c_\omega, \rho}\), 
    the solution \((z, \omega, v)\) is bounded such that :

    \begin{equation}
        \forall t \geq 0, (z(t), \omega(t), v(t)) \in \Delta_{c, c_\omega, \rho},
    \end{equation}

    where \(\lim_{t \to \infty} (z(t), \omega(t)) = 0\), and \(\forall t \geq t_0 > 0\):
    
    \begin{equation}
        s(z(t), \omega(t)) = 0,
    \end{equation}
    
    \begin{equation}
        v(t) = -\Delta(z(t), t).
    \end{equation}
\end{theorem}


The use of the scaling gain \(\mu\) gives the possibility to compensate the perturbation and ensure the 
stability of the closed loop system with explicite conditions. The question is, can we use the same method 
to estabilish an invariant set on the HOST algorithm with the Lyapunov function given in \cite{Laghrouche2017} ?
Is it sufficient to also just add the scaling gain \(\mu\) in the HOST algorithm to ensure the stability in 
case of algebraic loop ?



%=========================================================================================================%
%                                                                                                         %
%                                      NEW SECTION                                                       %
%                                                                                                         %
%=========================================================================================================%
\newpage
\section{Algebraic Loops in High Order Super Twisting}

As we saw in the previous section, the Super Twisting algorithm case can be used in case of time and state dependent
perturbation with the introduction of an algebraic loop can be extend to the HOST algorithm. The work is divided
in two steps because of the complexity in the reunion of the two theorems using different tools and methods.

First, we will try to get a first result on a evident case with an easy to handle class of perturbation to 
get the theory right. Then we will try to extend the result to a more general class of perturbation

\subsection{First step : taking an easy class of perturbation to handle}

We consider the system of dimension \(r \in \mathbb{N}\) with the following dynamics:

\begin{equation}
    \dot{z} = J_r z + (u + \Delta(z, t)) e_r
\end{equation}

where \(z \in \mathbb{R}^r\), \(J_r\) is the \(r\)-order Jordan block, \(e_r = [0, \ldots, 0, 1]^T \in \mathbb{R}^r\),
\(u \in \mathbb{R}\) is the control input and \(\Delta(z, t)\) is a perturbation depending on the state and time.

The perturbation satisfies the following assumption:

\begin{assumption}
    \label{assumption:perturbation_easy_case}
    There exists \(\delta, \delta_z, \delta_t > 0 \) such that for all \( t \geq 0 \) and \( z \in \mathbb{R}^r \):
    \begin{equation}
        \Delta(z, t) = \Delta_z(z) + \Delta_t(t)
    \end{equation}
    and,
    \begin{equation}
        \forall z, t \in \mathbb{R}^r \times \mathbb{R}_+, \\
        \begin{cases}
            |\Delta(z, t)| \leq \delta, \\
            \left| \frac{d\Delta_t(t)}{dt} \right| \leq \delta_t, \\
            \forall i \in \{1, \ldots, r\}, \left| \frac{\partial \Delta_z(z)}{\partial z_i} \right| \leq \delta_{z}
        \end{cases}
    \end{equation}
\end{assumption}

We assume that there exists the same stabilizing feedback law \(u_0\) and Lyapunov function \(V_1\) as in
Proposition \ref{prop:stabilizing_feedback} with \(p + (r+1)\kappa = 0\) and \(\alpha = 1 + \frac{\kappa}{2p_{r+1}} =1/2\).


The next assumption is the one that differentiate this section from the next one on the class of the perturbation.

\begin{assumption}
    \label{assumption:H4_extended}
    For all \(z \in \mathbb{R}^r\), we have : \(\frac{\partial \Delta_z(z)}{\partial z_r} = 0\)
\end{assumption}

This assumption is very strong and may be out of context because it is clear that it kills the algebraic loop :

\begin{equation}
    \dot{\Delta}(z, t) = \frac{d\Delta_t(t)}{dt} + \sum_{i=1}^{r} \frac{\partial \Delta_z(z)}{\partial z_i} \dot{z}_i = \frac{d\Delta_t(t)}{dt} + \sum_{i=1}^{r-1} \frac{\partial \Delta_z(z)}{\partial z_i} z_{i+1}
\end{equation}

Which doesn't contains the control input dynamics. We can state the following inequality independent 
of the control input:

\begin{equation}
    |\dot{\Delta}(z, t)| \leq \delta_t + \sum_{i=1}^{r-1} \delta_{z_i} |z_{i+1}| \leq \delta_t + \delta_z \|z\|_1
\end{equation}

And by placing us on the unit sphere and using the same notation as in the proof of 
Theorem in appendix~\ref{app:proof_theorem1} :

\begin{equation}
    |\dot{\Delta}(z, t)| \leq \delta_t + \sum_{i=1}^{r-1} \delta_{z_i} Z_{M_{i+1}} \leq \bar{\delta}
\end{equation}

with : \(Z_{M_i} = \max_{z \in R_p} |z_{i}|\)

But this case has an interest because even though \(\bar{\delta}\) doesn't depend on time and state,
the time scaling solution is no use here because of the state dependent part of the perturbation.

We consider the following HOST controller with a scaling gain \(\mu > 0\):
\begin{align}
    u_{ST}^{\mu}(t) &= \left(k_P u_0(z(t)) + k_I \xi^{\mu}_{\Delta}(t)\right) \\
    \dot{\xi}^{\mu}_{\Delta}(t) &= -\mu k_I \partial_r V_1(z(t)), \quad \xi^{\mu}_{\Delta}(0) = 0
\end{align}

Which leads to the closed loop dynamics:
\begin{equation}
    \begin{cases}
        \dot{z} &= J_r z + \left(k_P u_0(z) + k_I \xi^{\mu}_{\Delta}  \right) e_r \\
        \dot{\xi}^{\mu}_{\Delta} &= -\mu k_I \partial_r V_1(z(t)) + \dot{\Delta}(z, t), \quad \xi^{\mu}_{\Delta}(0) = \dot{\Delta}(0)
        \label{eq:closed_loop_dynamicNiclasHOST}
    \end{cases}
\end{equation}

The goal is to establish an invariant set constraining the system trajectories using the Lyapunov function 
\(W\) defined in \ref{eq:W}.

Let's consider the Lyapunov function \(W\) defined in \ref{eq:W} and the following set:
\begin{equation}
    \Lambda_{b, \mu} = \left\{ \begin{pmatrix} z \\ \xi \end{pmatrix} \mid W(z, \xi) \leq b \right\}
\end{equation}

We also cannot use the same approach as in \cite{Laghrouche2017} with the direct perturbed vector field.
The addition of the gain \(\mu\) has to be followed along the calculation to find a condition on it
that ensures the stability of the closed loop system.

The addition of the gain \(\mu\) in the HOST controller doesn't change the homogeneity of the closed loop system, so
we can stay on the unit sphere and compensate the perturbation with \(\mu\) like in the STA case by assuming this :

\begin{equation}
    \mu \leq 1 + \frac{\xi_{\max} \bar{\delta}}{(k_p - 1) V_{Umax}}
\end{equation}

where \(V_{Umax} = - \max_{z \in R_p} |\partial_r V_1(z) u_0(z)|\) and \(\xi_{\max} = \max_{\xi \in R_\xi} |\xi|\).

This assumption seems not very logical because it gives an upper bound on \(\mu\) but, we have to keep in mind
that \(\mu > 0\) to ensure the stability which creates a constraint on \(k_p\) that has to be sufficiently large to
ensure the existence of \(\mu\) and thus the stability of the closed loop system.

This assumption assures that the function \(W\) is a Lyapunov function for the closed loop system 
\ref{eq:closed_loop_dynamicNiclasHOST} and ensure that we can pursue the proof with the same method as in the 
last section with the STA using the invariant set \(\Lambda_{b, \mu}\) to establish the following theorem:

\begin{theorem}
    Consider the perturbed chain of integrators defined by the dynamics:
    \begin{equation}
    \dot{z} = J_r z + (u + \Delta(z, t)) e_r
    \end{equation}
    where the perturbation \(\Delta(z, t)\) satisfies Assumptions \ref{assumption:perturbation_easy_case} and \ref{assumption:H4_extended}. Assume there exists a continuous homogeneous feedback law \(u_0\) and a Lyapunov function \(V_1\) satisfying Proposition \ref{prop:stabilizing_feedback} with \(p + (r+1)\kappa = 0\) and \(\alpha = 1/2\).

    For every positive gains \(k_P \geq 1\) (that ensures the positiveness of \(\mu\)) 
    and \(k_I > 0\), and for a scaling gain \(\mu\) satisfying:

    \begin{equation}
        \mu \leq 1 + \frac{\xi_{\max} \bar{\delta}}{(k_p - 1) V_{Umax}}
    \end{equation}

    where \(V_{Umax} = - \max_{z \in R_p} |\partial_r V_1(z) u_0(z)|\) and \(\xi_{\max} = \max_{\xi \in R_\xi} |\xi|\), the dynamic state-feedback controller \(u_{ST}^{\mu}(\cdot)\) defined by:
    \begin{align}
        u_{ST}^{\mu}(t) &= \left(k_P u_0(z(t)) + k_I \xi^{\mu}_{\Delta}(t)\right) \\
        \dot{\xi}^{\mu}_{\Delta}(t) &= -\mu k_I \partial_r V_1(z(t)), \quad \xi^{\mu}_{\Delta}(0) = 0
    \end{align}
    ensures that the closed-loop system is finite-time stable with respect to the origin. 

\end{theorem}


\subsection{Second step : extending the result to a more general class of perturbation, main ideas}

I have not yet a complete result for this section but I will try to give the main ideas of the approach.

The main idea would be to consider the same system and controller as in the previous section but without
we can see quickly by following the same proof steps that the gain scaling \(\mu\) only in the integral part
of the HOST controller is not sufficient to compensate the perturbation derivative. A way should be to consider 
the new controller:

\begin{align}
    u_{ST}^{\mu}(t) &= \left(k_P u_0(z(t)) + \mu k_I \xi^{\mu}_{\Delta}(t)\right) \\
    \dot{\xi}^{\mu}_{\Delta}(t) &= -\mu k_I \partial_r V_1(z(t)), \quad \xi^{\mu}_{\Delta}(0) = 0
\end{align}

and try to mitigate the effects of the perturbation derivative with the gain \(\mu\) to keep the proprieties
of the Lyapunov function \(W\) defined in \ref{eq:W}. It would lead to the same conclusion.


Here the constion on the perturbation is different because we have to introduce the algebraic loop in its derivative:

\begin{align}
    \dot{\Delta}(z, t) &= \frac{d\Delta_t(t)}{dt} + \sum_{i=1}^{r} \frac{\partial \Delta_z(z)}{\partial z_i} \dot{z}_i \\
    &= \frac{d\Delta_t(t)}{dt} + \sum_{i=1}^{r-1} \frac{\partial \Delta_z(z)}{\partial z_i} z_{i+1} + \frac{\partial \Delta_z(z)}{\partial z_r} (k_P u_0(z) + \mu k_I \xi^{\mu}_{\Delta})
\end{align}

Here, we cannot find a bound \(\bar{\delta}\) independent of the state which complexifies a lot 
the calculations and the conditions on gain scaling \(\mu\) to ensure the stability of the closed loop system.

%=========================================================================================================%
%                                                                                                         %
%                                      NEW SECTION                                                       %
%                                                                                                         %
%=========================================================================================================%
\newpage
\section{Conclusion and future work}

Unfortunately, I didn't get enough time to finish the work on the HOST algorithm with algebraic loops
but I think the approach is correct and the first step is a good start to get the theory right. The next step
is to try to find gain scaling that fit the calculations to keep the Lyapunov function \(W\) as a Lyapunov 
function for the closed loop system. It is the easiest way to ensure the stability of the closed loop system that 
has the asset to stay close to the theory of \cite{tietze2025dynamic} with the ability to use the homogeneity
of the system to establish local invariant sets and extend the result to a global stability with finite time 
convergence.


The difficulty is to handle the unknown terms of the HOST controller that aren't very classical and 
create a condition on \( \mu \) that ensures the stability of the closed loop system while being 
possible because as we could have seen, the condition we can set on \( \mu \) in the first step have 
indirect repercussions on the choice of \( k_P \) and \( k_I \) that have to be sufficiently large to 
ensure the existence of \( \mu \) which increases the complexity of the tuning of the controller and the
work of the control input.

One last thing that I found very surprising is the lack of articles on the HOST algorithm especially on its
stability. The state of the art gives articles on its implementnation but few papers gives actual stability
proofs. The article \cite{Laghrouche2017} is the only one I found that gives a complete proof of the stability
of the HOST algorithm with a Lyapunov function that I could use and found very interesting to consider with
the work of \cite{tietze2025dynamic}.
