\chapter*{Acknowledgements}
\addcontentsline{toc}{chapter}{Acknowledgements}

I would like to express my deepest gratitude to Technische Universität 
Ilmenau for hosting me during my internship. Thanks to Prof. Johann Reger and
Niclas Tietze for their support throughout my project. I am also 
grateful to the members of the Institut für Automatisierungs und 
Systemtechnik for their collaboration and assistance. I thank the LS2N lab
for their financial support that allowed me to take part of the Jaime Moreno
and Leonid Friedman's course on Sliding Mode Control in last April.

I would like to thank École Centrale de Nantes for facilitating this international mobility 
experience. Additionally, I extend my appreciation to my colleagues and friends 
who made my stay in Ilmenau memorable and enriching. 

Lastly, I would like to thank my family and my friends in France for their 
support when I faced challenges and emotionally difficult moments during
this internship.

\chapter{Presentation of the Research Team and Scientific Context}


The Institut für Automatisierungs und Systemtechnik at Technische Universität Ilmenau is a dynamic research group specializing in advanced control systems and automation technologies. Led by Prof. Johann Reger, the team comprises 14 members, including professors, researchers, and PhD students, all contributing to various projects in the field of control engineering.


The primary research areas of the team include:
\begin{itemize}
    \item Diagnostic and prediction systems
    \item Control systems and management
    \item System identification
    \item Adaptive control methods
    \item Sliding mode control and variable structure systems
    \item Optimal robust control procedures
    \item Modulation-based estimation methods and FIR filters
    \item Process optimization
    \item Modeling and simulation of process engineering
    \item Development of algorithms for deterministic and stochastic optimization
    \item Simulation and optimization in production engineering
    \item Analysis and synthesis of environmental engineering installation networks
\end{itemize}

The team is known for its theoretical excellence and methodological contributions, collaborating closely with researchers from institutions such as UNAM in Mexico. The specific project I worked on involved the resolution of algebraic loops in sliding mode control approaches, focusing on state-dependent disturbances that create circular dependencies with the control signal.
