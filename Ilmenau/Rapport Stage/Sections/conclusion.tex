
\chapter{Personal Conclusion}

My internship at Technische Universität Ilmenau has been an enriching experience, both professionally and personally. Working within the Institut für Automatisierungs und Systemtechnik has provided me with valuable insights into advanced control systems and the practical challenges of implementing theoretical concepts in real-world scenarios.

On a professional level, I have gained significant experience in the field of sliding mode control and the resolution of algebraic loops. The collaborative environment and the expertise of my colleagues have greatly contributed to my understanding of complex control systems. I have also developed my research skills, including data analysis, simulation, and technical writing.

Culturally, living in Ilmenau has offered me a unique perspective on German life, particularly in the former East Germany. The blend of historical architecture and modern technological advancements at the university has been fascinating. The local culture, with its emphasis on efficiency and community, has left a lasting impression on me. Despite the occasional challenges with public transportation, such as the unreliable DeutschBahn, I have appreciated the simplicity and functionality of daily life in Ilmenau.

The academic environment at TU Ilmenau has been rigorous and focused on practical applications, which has enhanced my problem-solving skills and prepared me for future challenges in the field of control engineering. The interactions with my colleagues and the opportunity to work on cutting-edge research projects have been incredibly rewarding.

In conclusion, this internship has not only advanced my technical knowledge but also broadened my cultural horizons. It has reinforced my interest in pursuing a career in control systems and has provided me with valuable experiences that will benefit my future endeavors. I am grateful for the opportunity to have been a part of such a dynamic and innovative research team.
%======================= CONCLUSION =============================
\chapter*{Conclusion}
\addcontentsline{toc}{chapter}{Conclusion}

In the end, this internship has been on to very differnt topics. On one hand, the MFC I had to work on 
to implement it on a simulation and a real system. On the other hand, the algebraic loops which I would 
have like to go deeper but the time was not enough to do so.

The MFC is a very interesting control strategy, which has many assets compared to high gain control. And 
above all, its simplified definition is very quick to implement and quite independent of system parameters as 
it is used on linearized systems. In our case, we could extend the state to make the internal dynamics vanish but
as showed in \cite{Willkomm2023MFC}, if they are input-to-state stable, the MFC is totally applicable.
Its robustness and assets make it a very good candidate for industrial applications.

The algebraic loops resolution in sliding mode control approaches is a very interesting topic, which
introduced me to the scientific research in the filed of stability proofs and the real challenges 
of Lyapunov-based proofs. The method proposed in \cite{Laghrouche2017} is a good start for a certain class of
perturbations but the general case is still open. 

Finally, I regret not having been able to go deeper in this topic, as I would have liked to. And I want 
to acknowledge that it can be very important to state the conditions of the internship because some 
miscommunications led to a lack of supervising and I had to work mainly on my own which was sometimes challenging.
Especially when I had to work on the MFC topic which was not my main topic of the internship.


I'll also remember the cultural experience in Ilmenau, a small town in the former East Germany. The trip I could do 
during my free time in this country rich in history and culture. I could also learn again some German 
which was very nice and interesting.
