\chapter{Proof of Theorem 1 (Lagrouche et al., 2017)}
\label{app:proof_theorem1}
This proof is not given in the original paper \cite{Laghrouche2017},
We introduce the positive definite and homogeneous function of degree \(2p_{r+1}\):
\begin{equation}
V(z, \xi) = V_1 + \frac{1}{2} \xi^2
\end{equation}
Calculating the time derivative:
\begin{equation}
\dot{V} = \dot{V}_1 + \xi \dot{\xi}
\end{equation}
\begin{equation}
= \dot{V}_1 - k_I \xi \partial_r V_1(z)
\end{equation}
where \(k_I \xi = \dot{z}_r - k_p u_0(z)\)
Thus,
\begin{equation}
\dot{V} = \dot{V}_1 + (k_p u_0(z) - \dot{z}_r) \partial_r V_1(z)
\end{equation}
\begin{equation}
= \dot{V}_1 + k_p \partial_r V_1(z) u_0 - \dot{z}_r \partial_r V_1(z)
\end{equation}

\begin{equation}
= \dot{V}_1 + (k_p - 1) \partial_r V_1(z) u_0 \leq -c V_1^\alpha
\end{equation}


We define the final candidate Lyapunov function \(L\):
\begin{equation}
W(z, \xi) = V(z, \xi)^{2-\alpha} - A z_r \xi
\end{equation}
where \(W\) is smooth except at the origin and homogeneous regarding the function \(\Psi(z, \xi)\) of degree
\begin{equation}
2(2-\alpha) p_{r+1} = p_r + p_{r+1}, \quad W > 0 \text{ because } V > 0 \text{ for a small enough value of } A
\end{equation}


Moreover, : 

\begin{align*}
    \dot{W}(z, \xi) &= (2-\alpha) V^{1-\alpha} \dot{V} - A (\dot{z}_r \xi + z_r \dot{\xi}) \\
    &= (2-\alpha) V^{1-\alpha} \dot{V} - A (k_p u_0 + k_I \xi) \xi + A z_r k_i \partial_r V_1(z)
\end{align*}


and \(\dot{V} \leq -c V_1^\alpha\), Thus,
\begin{equation}
\dot{W} \leq -c (2-\alpha) V^{1-\alpha} V_1^\alpha - A k_p u_0 \xi + A k_I \xi \partial_r V_1 - A k_I \frac{\xi^2}{2}
\end{equation}


Since \(V \geq V_1\) and \(2 - \alpha > 0\),
\begin{equation}
V^{1-\alpha} \geq V_1^{1-\alpha} > 0
\end{equation}
Thus,
\begin{equation}
(2-\alpha) V^{1-\alpha} \geq V_1^{1-\alpha}
\end{equation}
Therefore,
\begin{equation}
- c (2-\alpha) V^{1-\alpha} V_1^\alpha \leq -c V_1
\end{equation}
Which gives:
\begin{equation}
\dot{W} \leq -c V_1 + A k_I |z_r \partial_r V_1(z)| - A k_p u_0 \xi - A k_I \xi^2
\end{equation}

Knowing that:
\begin{equation}
- A k_p u_0 \xi - A k_I \xi^2 \leq A k_p^2 \frac{u_0^2}{k_I} - \frac{1}{2} A k_I \xi^2
\end{equation}
Thus, we can write:


\begin{equation}
\dot{W} \leq -c V_1 + A k_p^2 \frac{u_0^2}{k_I} + A k_I |z_r \partial_r V_1(z)| - \frac{1}{2} A k_I \xi^2
\end{equation}


It is easy to prove that:
\begin{equation}
z \rightarrow -c V_1 + A k_p^2 \frac{u_0^2}{k_I} + A k_I |z_r \partial_r V_1(z)| 
\end{equation}
is homogeneous of degree \(2 p_{r+1}\) with respect to \((\delta_{\epsilon})_{\epsilon > 0}\).


Let \(R_p\) and \(N_p\) be the unit balls of \(\mathbb{R}^r\) and \(\mathbb{R}^{r+1}\) associated with the weights \(p_i\) for \(i \in [1, r]\) and \(i \in [1, r+1]\) respectively.
Let :
\begin{equation}
V_m = \min_{(z, \xi) \in N_p} V(z, \xi)^{2-\alpha}
\end{equation}
\begin{equation}
Z_m = \max_{(z, \xi) \in N_p} |z_r \xi|
\end{equation}
\begin{equation}
Z_m^1 = \max_{z \in R_p} |z_r \partial_r V_1(z)|
\end{equation}
\begin{equation}
Z_m^2 = \max_{z \in R_p} u_0^2
\end{equation}
\begin{equation}
V_m^1 = \min_{z \in R_p} V_1
\end{equation}

Since \(V > 0\), \(V_m > 0\), we can always choose \(A\) such that:
\begin{equation}
A \leq \min \left( \frac{V_m}{2 Z_m}, \frac{c k_I V_m^{1}}{2 \left( k_p^2 Z_m^2 + k_I^2 Z_m^2 \right)}, \frac{c}{2 k_I} \right)
\end{equation}
Then,
\begin{equation}
A \leq \frac{V_m}{2 Z_m} \implies A Z_m < V_m
\end{equation}
Since \(A z_r \xi \leq A Z_m < V_m\),
\begin{equation}
- A z_r \xi > -V_m
\end{equation}
Thus,
\begin{equation}
W = V^{2-\alpha} - A z_r \xi > V^{2-\alpha} - V_m
\end{equation}
Since \(V^{2-\alpha} \geq V_m\), it follows that:
\begin{equation}
V^{2-\alpha} - V_m \geq 0 \implies W > 0
\end{equation}

\begin{equation}
\dot{W} \leq -\frac{1}{2} \left( c V_1 + A k_I \xi^2 \right) + A \left( \frac{k_p^2 u_0^2}{k_I} + k_I |z_r \partial_r V_1| \right) - \frac{1}{2} c V_1
\end{equation}
Since \(0 \leq \frac{1}{2} \frac{k_p^2 u_0^2}{k_I} + k_I |z_r \partial_r V_1| \leq \frac{k_p^2 Z_m^2}{k_I} + k_I Z_m^1\),
\begin{equation}
0 \leq A \leq \frac{c k_I V_m^{1}}{2 \left( \frac{k_p^2 Z_m^2}{k_I} + k_I^2 Z_m^2 \right)}
\end{equation}
Thus,
\begin{equation}
A \left( \frac{k_p^2 u_0^2}{k_I} + k_I |z_r \partial_r V_1| \right) \leq \frac{c k_I V_m^{1}}{2k_I} = \frac{c V_m^1}{2}
\end{equation}


Hence,
\begin{equation}
\dot{W} \leq -\frac{1}{2} \left( c V_1 + A k_I \xi^2 \right)
\end{equation}

The homogeneity of \(W\) implies the global validity of the inequality.


Let's show that:
\begin{equation}
\frac{1}{2} V^{2-\alpha} \leq W \leq \frac{3}{2} V^{2-\alpha}
\end{equation}


\begin{equation}
W = V^{2-\alpha} - A z_r \xi
\end{equation}
\begin{equation}
= \frac{1}{2} V^{2-\alpha} + \left( \frac{1}{2} V^{2-\alpha} - A z_r \xi \right)
\end{equation}

Since \(A \leq \frac{V_m}{2 Z_m}\),
\begin{equation}
A Z_m \leq \frac{1}{2} V_m \leq \frac{1}{2} V^{2-\alpha}
\end{equation}
Thus, \(A Z_m \geq A z_r \xi\),
\begin{equation}
\frac{1}{2} V^{1-\alpha} - A z_r \xi \geq 0
\end{equation}
Therefore,
\begin{equation}
W \geq \frac{1}{2} V^{2-\alpha}
\end{equation}

On the same way,
\begin{equation}
\dot{W} = \frac{3}{2} V^{2-\alpha} - \left( \frac{1}{2} V^{2-\alpha} + A z_r \xi \right) 
\end{equation}
Since,
\begin{equation}
\dot{W} \leq \frac{3}{2} V^{2-\alpha}
\end{equation}
Therefore, we have:
\begin{equation}
\frac{1}{2} V^{2-\alpha} \leq W \leq \frac{3}{2} V^{2-\alpha}
\end{equation}

Now we have to show that:
\begin{equation}
\dot{W} \leq -A k_I V
\end{equation}
We know that:
\begin{equation}
\dot{W} \leq -\frac{(c V_1 + A k_I \xi^2)}{2} \implies \dot{W} \leq -\frac{c}{2} V_1 - \frac{A k_I}{2} \xi^2
\end{equation}
Given \(V = V_1(z) + \frac{1}{2} \xi^2 \implies V > V_1\),


\begin{equation}
A \leq \frac{c}{2 k_I} \implies \frac{c}{2 A k_I} \geq 1
\end{equation}
\begin{equation}
\implies \frac{c}{2 A k_I} V_1 + \frac{1}{2} \xi^2 \geq V_1 + \frac{1}{2} \xi^2
\end{equation}
\begin{equation}
\implies -A k_I \left[ \frac{c}{2 A k_I} V_1 + \frac{1}{2} \xi^2 \right] \leq -A k_I \left[ V_1 + \frac{1}{2} \xi^2 \right]
\end{equation}
Therefore, it follows that:
\begin{equation}
\dot{W} \leq -A k_I V
\end{equation}


We have the two inequalities:
\begin{align*}
    &\frac{1}{2} V^{2-\alpha} \leq W \leq \frac{3}{2} V^{2-\alpha} \quad (1) \\
    &\dot{W} \leq -A k_I V \quad (2)
\end{align*}

So we can write the final inequality as:
\begin{equation}
\dot{W} \leq -d W^{\frac{1}{2-\alpha}}
\end{equation}
where \(d = \frac{A k_I}{4}\) (since \(2-\alpha > 1\)).
Let \(f(x) = x^{\frac{1}{2-\alpha}}\), which is an increasing function.
Thus,
\begin{equation}
\left(\frac{1}{2}\right)^{\frac{1}{2-\alpha}} V \leq W^{\frac{1}{2-\alpha}} \leq \left(\frac{3}{2}\right)^{\frac{1}{2-\alpha}} V
\end{equation}

\begin{align*}
&\text{Since } 2 - \alpha > 1 \implies \frac{1}{2-\alpha} < 1 \\
&\text{If } \frac{1}{2} \leq 1 \implies \left(\frac{1}{2}\right)^{\frac{1}{2-\alpha}} > \frac{1}{2} \implies \frac{1}{2} V^ \leq \left(\frac{1}{2}\right)^{\frac{1}{2-\alpha}} V \\
&\text{If } \frac{3}{2} > 1 \implies \left(\frac{3}{2}\right)^{\frac{1}{2-\alpha}} < \frac{3}{2} \implies \frac{3}{2} V \leq  \left(\frac{3}{2}\right)^{\frac{1}{2-\alpha}} V
\end{align*}
By sandwich theorem:
\begin{equation}
\frac{1}{2} V\leq W^{\frac{1}{2-\alpha}} \leq \frac{3}{2} V
\end{equation}
Therefore:
\begin{equation}
-\dot{W} \geq A k_I V \implies -\frac{1}{A k_I}\dot{W} \geq V \implies -\frac{3 A k_I}{2} \dot{W} \geq \frac{3}{2} V \geq W^{\frac{1}{2-\alpha}}
\end{equation}

Hence,
\begin{equation}
\dot{W} \leq -\frac{2 A k_I}{3} W^{\frac{1}{2-\alpha}}
\end{equation}
\begin{align}
    \frac{2}{3} > \frac{1}{4} \implies -\frac{2}{3} < -\frac{1}{4}
\end{align}

In the end:
\begin{equation}
\dot{W} \leq -\frac{A k_I}{4} W^{\frac{1}{2-\alpha}}
\end{equation}
This concludes the proof of the Theorem 

