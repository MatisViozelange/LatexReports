\documentclass[12pt]{report}


% ====================== PACKAGES ======================
\usepackage[french]{babel} % French language support
\usepackage[utf8]{inputenc} % UTF-8 encoding
\usepackage[T1]{fontenc} % Character encoding
\usepackage{amsmath, amssymb, amsfonts} % Math packages
\usepackage{graphicx} % Required for inserting images
\usepackage{float} % Improved handling of floating elements
\usepackage{hyperref} % For hyperlinks
\usepackage{array, tabularx, multirow, multicol} % Table packages
\usepackage{caption, subcaption} % Caption and subcaption for figures and tables
\usepackage{setspace} % Set line spacing
\usepackage{abstract} % Abstract layout
\usepackage{color, xcolor} % Color handling
\usepackage{lipsum} % For generating filler text
\usepackage{fancyhdr} % Fancy headers and footers
\usepackage{titlesec} % Section formatting
\usepackage{enumerate} % Enumerate with custom labels
\usepackage{booktabs, colortbl} % Improved table formatting
\usepackage{geometry} % To define page margins
\usepackage{longtable} % For tables spanning multiple pages
\usepackage{listings}
\usepackage{bm}

% ====================== COMMANDS ======================
\providecommand{\abs}[1]{\lvert#1\rvert} % Absolute value command
\DeclareMathOperator{\sign}{sign} % Sign function

% ====================== PAGE GEOMETRY ======================
\geometry{top=2cm, bottom=3cm, left=2cm, right=2cm} % Adjust top margin to provide space for header
\setlength{\headheight}{45pt} % Increase headheight to fit header content
\setlength{\headsep}{20pt} % Add space between header and text

% ====================== PDF METADATA ======================
\hypersetup{ % Information about the PDF document
    pdfauthor = {Premier Auteur}, % Authors
    pdftitle = {Nom du Projet - Sujet du Projet}, % Title
    pdfsubject = {Mémoire de Projet}, % Subject
    pdfkeywords = {Tag1, Tag2, Tag3}, % Keywords
    pdfstartview={FitH}, % Adjust page width to screen width
    colorlinks=true, % Colored links
    citecolor=black, % Citation links color
    filecolor=black, % File links color
    linkcolor=black, % Internal links color
    urlcolor=black % URL links color
}

\lstset{
    language=Python,
    basicstyle=\ttfamily\footnotesize,
    keywordstyle=\color{blue},
    commentstyle=\color{orange},
    stringstyle=\color{red},
    showstringspaces=false,
    numberstyle=\tiny\color{gray},
    numbers=left,
    stepnumber=1,
    numbersep=5pt,
    frame=single,
    breaklines=true,
    backgroundcolor=\color{lightgray},
    captionpos=b,
    tabsize=4,
    morekeywords={Material, self},
}

% ====================== HEADER AND FOOTER ======================
\pagestyle{fancy} % Enable fancy headers and footers
\fancyhf{} % Clear all header and footer fields

\fancyhead[L]{ % Left header
    \begin{minipage}{1.5cm}
        \centering
        \includegraphics[width=0.80\textwidth]{imgs/LogoCN_Q.pdf}
    \end{minipage}
    \begin{minipage}{12cm}
        \small \textsc{École}\\
        \small \textsc{Centrale}\\
        \small \textsc{Nantes}\\
    \end{minipage}
}
\fancyhead[R]{ % Right header
    \begin{minipage}{4.8cm}
        \raggedleft
        \small \textsc{stage vision}\\
    \end{minipage}
}
\fancyfoot[R]{\large \textbf{\thepage}} % Right footer with page number

\renewcommand{\headrulewidth}{0.2pt} % Header rule width
\renewcommand{\footrulewidth}{0.2pt} % Footer rule width

% ====================== SECTION FORMATTING ======================
\titleformat{\chapter}[display]
    {\normalfont\bfseries}{}{0pt}{\Large}
\titlespacing*{\chapter}{0pt}{-20pt}{20pt}

% ====================== BEGIN DOCUMENT ======================
\begin{document}
\pagenumbering{arabic} % Start page numbering in arabic numerals
\renewcommand{\labelenumii}{\arabic{enumi}.\arabic{enumii}}

% Title page
\begin{titlepage}
    \centering
    \vspace*{0cm}
    \includegraphics[width=0.3\textwidth]{imgs/LogoCN_Q.pdf}
    
    \vspace{3cm}
    
    \Huge\textbf{Rapport TP3 : ACSIP}\\
    \vspace{1cm}
    \Large\textbf{Création Sonore}\\
    
    \vspace{2cm}
    \Large\textbf{Design : message sonore de validité}\\
    
    \vfill
    
    \Large{Viozelange Matis}\\
    \vspace{0.5cm}
    
    \vfill

    \vspace{2cm}
    
    \Large\textit{date : \today}\\
    
    \vspace{0.5cm}
    
    \Large\textbf{École Centrale Nantes}\\
    
    \vspace*{1cm}
\end{titlepage}


\newpage
\thispagestyle{empty}
~

\chapter{Réflexion sur le design sonore}

\section{Problématique}

\section{Réflexion}

Pour la conception globale du son, j'ai en tête une petite mélodie numérique de trois notes. %
Les deux premières notes seraient identiques et la troisième donnerait le ton de la validité. En effet, %
On se rend facilement compte qu'un intervalle ascendant est synonime de bonne nouvelle et inversement. % 
Pour donner un coté agréable au son, il faut trouver un rythme entrainant. Rapprocher les deux dernière notes %
donnerait un coté sautillant à la mélodie. Nous opterons pour un rythme comme sur la figure \textbf{A FAIRE}.%

Enfin, toujours dans l'optique de rendre agréable le son, il faudrait lui donner un envellope qui %
le rendrait le moins agessif possible. Plutôt quelque chose de rond sans attaque.

\begin{itemize}
    \item Compresseur
    \item Soft overdrive
    \item Fade in et out 
\end{itemize}

\chapter{Réalisation}

\end{document}